%%%%%%%%%%%%%%%%%%%%%%%%%%%%%%%%%%%%%%%%%
% fphw Assignment
% LaTeX Template
% Version 1.0 (27/04/2019)
%
% This template originates from:
% https://www.LaTeXTemplates.com
%
% Authors:
% Class by Felipe Portales-Oliva (f.portales.oliva@gmail.com) with template 
% content and modifications by Vel (vel@LaTeXTemplates.com)
%
% Template (this file) License:
% CC BY-NC-SA 3.0 (http://creativecommons.org/licenses/by-nc-sa/3.0/)
%
%%%%%%%%%%%%%%%%%%%%%%%%%%%%%%%%%%%%%%%%%

%----------------------------------------------------------------------------------------
%	PACKAGES AND OTHER DOCUMENT CONFIGURATIONS
%----------------------------------------------------------------------------------------


\documentclass[
12pt, % Default font size, values between 10pt-12pt are allowed
%letterpaper, % Uncomment for US letter paper size
%spanish, % Uncomment for Spanish
]{fphw}
% Template-specific packages
\usepackage[utf8]{inputenc} % Required for inputting international characters
\usepackage[T1]{fontenc} % Output font encoding for international characters
\usepackage{mathpazo} % Use the Palatino font

\usepackage{graphicx} % Required for including images

\usepackage{booktabs} % Required for better horizontal rules in tables

\usepackage{listings} % Required for insertion of code

\usepackage{enumerate} % To modify the enumerate environment

\usepackage{float} %设置图片浮动位置的宏包
\usepackage{subfigure} %插入多图时用子图显示的宏包
\usepackage{ctex}
\usepackage{amsmath}
\usepackage{mathrsfs}

\usepackage{listings}
\usepackage{color}

\definecolor{dkgreen}{rgb}{0,0.6,0}
\definecolor{gray}{rgb}{0.5,0.5,0.5}
\definecolor{mauve}{rgb}{0.58,0,0.82}

\lstset{frame=tb,
	language=Python,
	aboveskip=3mm,
	belowskip=3mm,
	showstringspaces=false,
	columns=flexible,
	basicstyle={\small\ttfamily},
	numbers=none,
	numberstyle=\tiny\color{gray},
	keywordstyle=\color{blue},
	commentstyle=\color{dkgreen},
	stringstyle=\color{mauve},
	breaklines=true,
	breakatwhitespace=true,
	tabsize=3
}
%----------------------------------------------------------------------------------------
%	ASSIGNMENT INFORMATION
%----------------------------------------------------------------------------------------

\title{Homework \#5} % Assignment title

\author{Chengqi Liu (1954148), Maitraiyi Dandekar (1990136),\\ Zakariae Jabbour (2039702)} % Student name

\date{October 10th, 2023} % Due date

\institute{Eindhoven University of Technology} % Institute or school name

\class{Cryptology (2MMC10)} % Course or class name

\professor{Tanja Lange} % Professor or teacher in charge of the assignment

%----------------------------------------------------------------------------------------

\begin{document}

\maketitle % Output the assignment title, created automatically using the information in the custom commands above

%----------------------------------------------------------------------------------------
%	ASSIGNMENT CONTENT
%----------------------------------------------------------------------------------------
\textbf{1.\\}
$p=1249$. $p-1=2^5*3*13$. $g=7$. $h_b=1195$.\\
Let
\begin{align*}
	b=&a_0+a_12+a_22^2+a_32^3+a_42^4\\
	=&\sum_{j=0}^{4}a_j2^j,\  (a_j\in\mathbb{Z}_2) \mod 2^5
\end{align*}
$\forall r\in\{1,2,3,4,5\}$:
\begin{align*}
	&g^b=h_b\mod p\\
	\Rightarrow&(g^b)^{\frac{p-1}{2^r}}=h_b^{\frac{p-1}{2^r}}\mod p\\
	\Rightarrow&g^{\sum_{j=0}^{4}2^{j-r}(p-1)a_j}=h_b^{\frac{p-1}{2^r}}\mod p\\
	\Rightarrow&\prod_{j=0}^{4}g^{2^{j-r}(p-1)a_j}=h_b^{\frac{p-1}{2^r}}\mod p\\
	\Rightarrow&\prod_{j=0}^{r-1}g^{\frac{p-1}{2^{r-j}}a_j}\prod_{j=r}^{4}g^{2^{j-r}(p-1)a_j}=h_b^{\frac{p-1}{2^r}}\mod p
\end{align*}
When $j\geq r$, because $g^{p-1}=1\mod p$, 
\[g^{2^{j-r}(p-1)a_j}=1\mod p\]
So,
\[\prod_{j=r}^{4}g^{2^{j-r}(p-1)a_j}=1 \mod p\]
So,
\begin{align}
	&\prod_{j=0}^{r-1}g^{\frac{p-1}{2^{r-j}}a_j}=h_b^{\frac{p-1}{2^r}}\mod p\\
	\Rightarrow&(g^{\frac{p-1}{2}})^{a_{r-1}}=\frac{h_b^{\frac{n-1}{2^r}}}{\prod_{j=0}^{r-2}g^{\frac{p-1}{2^{r-j}}a_j}} \mod p
\end{align}
Formula (2) is the recursive formula of $a_j$. We can just solve this DLP (using BSGS) and get $a_{r-1}$ from $a_0, a_1,...,a_{r-2}$. The order of $g^{\frac{p-1}{2}}$ is 2.\\
Let $r=1$ in formula (1), we have
\[(g^\frac{p-1}{2})^{a_0}=h_b^\frac{p-1}{2}  \mod 2\]
Solve this DLP using BSGS and we can get the initial value $a_0$.
Run the program "Pohlig-Hellman.py" and the first part of output is:\\
a0 is 0\\
a1 is 1\\
a2 is 0\\
a3 is 0\\
a4 is 1\\
So,
\[b=0+2+0+0+2^4=18\mod 2^5\]
For 3 and 13, we have:
\begin{align*}
	&(g^\frac{p-1}{3})^b=h_b^\frac{p-1}{3} \mod p\\
	\Rightarrow&(7^{416})^b=1195^{416} \mod p\\
	\Rightarrow&1155^b=1155 \mod p\\
	&(g^\frac{p-1}{13})^b=h_b^\frac{p-1}{13} \mod p\\
	\Rightarrow&(7^{96})^b=1195^{96} \mod p\\
	\Rightarrow&994^b=240 \mod p
\end{align*}
Run the program "Pohlig-Hellman.py" and the second part of output is:\\
log\_1155(1155) is 1\\
log\_994(240) is 12\\
So,
\begin{align*}
	b&=18 \mod 2^5\\
	b&=1 \mod 3\\
	b&=12 \mod 13
\end{align*}
Using CRT to solve this equation set. Run the program "Pohlig-Hellman.py".
\begin{lstlisting}
	ans=Pohlig_Hellman(g,hb,p-1,p_l,e_l,p)[0]
	print("The answer is",ans)
	if pow(g,ans,p)==hb:
	print("The answer is correct.")
	else:
	print("The answer is wrong!")
\end{lstlisting}
The third part of output is:\\
The answer is 1234\\
The answer is correct.\\
So, $b=1234$ and $g^b=1195=h_b$. You can find all the codes in file "Pohlig-Hellman.py".\\
\textbf{2.(a)}\\
$n=396553$. $e=17$. $p=541$. $q=733$.\\
\begin{align*}
	\phi(n)=&(p-1)(q-1)=395280\\
	d=&e^{-1}\mod \phi(n)=302273\\
	d_p=&d\mod p-1=413\\
	d_q=&d\mod q-1=689\\
	u=&p^{-1} \mod q=691
\end{align*}
So, $(n,p,q,d_p,d_q,u)=(396553,541,733,413,689,691)$.\\
\textbf{(b)}\\
$a=2$. $1\leq a<p$.
\begin{align*}
	a^p \mod p=2^{541}\mod 541=2=a
\end{align*}
So, $p$ passes the one-round Fermat test. It's probably a prime.\\
\textbf{(c)}\\
$c=234040$. Using CRT method, we have
\begin{align*}
	c_p=&c \mod p=328\\
	c_q=&c\mod q=213\\
	m_p=&c_p^{d_p}\mod p=37\\
	m_q=&c_q^{d_q}\mod q=162\\
	m=&m_p+pu(m_q-m_p)\mod n\\
	=&37+541*691*(162-37) \mod 396553\\
	=&332211
\end{align*}
Verify it:
\begin{align*}
	c'=m^e \mod n=234040=c
\end{align*}
So, the answer is correct.



\end{document}

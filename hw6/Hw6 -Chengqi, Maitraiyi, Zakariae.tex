%%%%%%%%%%%%%%%%%%%%%%%%%%%%%%%%%%%%%%%%%
% fphw Assignment
% LaTeX Template
% Version 1.0 (27/04/2019)
%
% This template originates from:
% https://www.LaTeXTemplates.com
%
% Authors:
% Class by Felipe Portales-Oliva (f.portales.oliva@gmail.com) with template 
% content and modifications by Vel (vel@LaTeXTemplates.com)
%
% Template (this file) License:
% CC BY-NC-SA 3.0 (http://creativecommons.org/licenses/by-nc-sa/3.0/)
%
%%%%%%%%%%%%%%%%%%%%%%%%%%%%%%%%%%%%%%%%%

%----------------------------------------------------------------------------------------
%	PACKAGES AND OTHER DOCUMENT CONFIGURATIONS
%----------------------------------------------------------------------------------------


\documentclass[
12pt, % Default font size, values between 10pt-12pt are allowed
%letterpaper, % Uncomment for US letter paper size
%spanish, % Uncomment for Spanish
]{fphw}
% Template-specific packages
\usepackage[utf8]{inputenc} % Required for inputting international characters
\usepackage[T1]{fontenc} % Output font encoding for international characters
\usepackage{mathpazo} % Use the Palatino font

\usepackage{graphicx} % Required for including images

\usepackage{booktabs} % Required for better horizontal rules in tables

\usepackage{listings} % Required for insertion of code

\usepackage{enumerate} % To modify the enumerate environment

\usepackage{float} %设置图片浮动位置的宏包
\usepackage{subfigure} %插入多图时用子图显示的宏包
\usepackage{ctex}
\usepackage{amsmath}
\usepackage{mathrsfs}

\usepackage{listings}
\usepackage{color}

\definecolor{dkgreen}{rgb}{0,0.6,0}
\definecolor{gray}{rgb}{0.5,0.5,0.5}
\definecolor{mauve}{rgb}{0.58,0,0.82}

\lstset{frame=tb,
	language=Python,
	aboveskip=3mm,
	belowskip=3mm,
	showstringspaces=false,
	columns=flexible,
	basicstyle={\small\ttfamily},
	numbers=none,
	numberstyle=\tiny\color{gray},
	keywordstyle=\color{blue},
	commentstyle=\color{dkgreen},
	stringstyle=\color{mauve},
	breaklines=true,
	breakatwhitespace=true,
	tabsize=3
}
%----------------------------------------------------------------------------------------
%	ASSIGNMENT INFORMATION
%----------------------------------------------------------------------------------------

\title{Homework \#6} % Assignment title

\author{Chengqi Liu (1954148), Maitraiyi Dandekar (1990136),\\ Zakariae Jabbour (2039702)} % Student name

\date{October 17th, 2023} % Due date

\institute{Eindhoven University of Technology} % Institute or school name

\class{Cryptology (2MMC10)} % Course or class name

\professor{Tanja Lange} % Professor or teacher in charge of the assignment

%----------------------------------------------------------------------------------------

\begin{document}

\maketitle % Output the assignment title, created automatically using the information in the custom commands above

%----------------------------------------------------------------------------------------
%	ASSIGNMENT CONTENT
%----------------------------------------------------------------------------------------
\textbf{1.\\}
The conditions are:
\begin{align*}
	n=&pq=7684607040813031964568123727442263397500506224420545927139570285341\\
	p=&a+r\\
	a=&3855587076697238083701498334674944\\
	r<&2^{36}
\end{align*}
Let $F(r)=(r+a)$, then $r$ is a solution of $F(r)\equiv0\mod p$. Let $X=2^{36}$. $r<X$. Construct the lattice:
\[B=
\begin{bmatrix}
	n & 0 & 0 & 0\\
	a & X & 0 & 0\\
	0 & aX & aX^2 & 0\\
	0 & 0 & aX^2 & aX^3\\
\end{bmatrix}
\]
You can find our codes in "LLL.py".\\
Run our codes and get the outputs.\\
(The matrix after LLL reduction is too large so we omit it here. You can run our program to find it.)\\
The function of $r$ is:
\begin{align*}
	F(r)=&-96555903568429351669384558377410+294955517584034757255986690392064r\\
	&+60298211352416430157901672218624r^2+0r^3
\end{align*}
Solve $F(r)$ by Newton's method and check whether $n\equiv 0 \mod a+r$. The outputs are:\\
$r$ is: 21163420865\\
The answer is correct.\\
Then, find $p=a+r$ and $q=n/(a+r)$:\\ 
$p$ is: 3855587076697238083701519498095809\\
$q$ is: 1993109450765095312123565166297088\\\\
\textbf{2.\\}
(In case of symbol confusion, we use $N=pq$ in this part.)\\
The Howgrawe-Graham Theorem is: There is some $x_0$ such that $|x_0|\leq X$ and$F(x_0)\equiv0 \mod M$. $b_F=(a_0,a_1X,...,a_dX^d)$. If $||b_F||<\frac{M}{\sqrt{d+1}}$, then $F(x_0)=0$.\\
And, by the property of LLL lattice basis reduction, we have
\[||b_F||\leq2^{\frac{n-1}{4}}\det(B)^\frac{1}{n}\]
So, we can solve $r$ if
\begin{align*}
	&2^{\frac{n-1}{4}}\det(B)^\frac{1}{n}<\frac{M}{\sqrt{d+1}}
\end{align*}
We can construct the Lattice with $N^h,N^{h-1}F(x),...,F^h(x),xF^h(x),x^2F^h(x),...,x^{k-h}F^h(x)$.
where $n=k$, $d=k-1$, $\det(B)=N^{\frac{(h+1)h}{2}}X^\frac{(k+1)k}{2}$, $M=p^h$.
\begin{align*}
	&2^{\frac{n-1}{4}}\det(B)^\frac{1}{n}<\frac{M}{\sqrt{d+1}}\\
	\Leftrightarrow&2^{\frac{k-1}{4}}(N^{\frac{(h+1)h}{2}}X^\frac{(k+1)k}{2})^\frac{1}{k}<\frac{p^h}{\sqrt{k}}\\
	\Leftrightarrow&X<(\frac{p^h}{\sqrt{k}}2^{-\frac{k-1}{4}}N^{-\frac{(h+1)h}{2k}})^\frac{2}{k+1}
\end{align*}
We can just let $h=4$, $k=8=2^3$, then it becomes
\[X<(p^42^{-\frac{13}{4}}N^{-\frac{5}{4}})^\frac{2}{9}\]
In the previous exercise, we have
\begin{align*}
	X=2^{36}=&68719476736\\
	(p^42^{-\frac{13}{4}}N^{-\frac{5}{4}})^\frac{2}{9}\approx&114170204032
\end{align*}
So, $X<(p^42^{-\frac{13}{4}}N^{-\frac{5}{4}})^\frac{2}{9}$ is correct, which means $r$ is small enough. We can solve $r$ and factor $N$ in polynomial time. \\
(Note that we didn't use the very complicated Lattice with $h=4$ $k=8$, because $||b_F||\leq2^{\frac{n-1}{4}}\det(B)^\frac{1}{n}$ only gives an extreme worst-case, and the  Lattice with $h=4$ $k=8$ is very hard to compute. Usually a simpler Lattice can also give a correct result.)
\end{document}

%%%%%%%%%%%%%%%%%%%%%%%%%%%%%%%%%%%%%%%%%
% fphw Assignment
% LaTeX Template
% Version 1.0 (27/04/2019)
%
% This template originates from:
% https://www.LaTeXTemplates.com
%
% Authors:
% Class by Felipe Portales-Oliva (f.portales.oliva@gmail.com) with template 
% content and modifications by Vel (vel@LaTeXTemplates.com)
%
% Template (this file) License:
% CC BY-NC-SA 3.0 (http://creativecommons.org/licenses/by-nc-sa/3.0/)
%
%%%%%%%%%%%%%%%%%%%%%%%%%%%%%%%%%%%%%%%%%

%----------------------------------------------------------------------------------------
%	PACKAGES AND OTHER DOCUMENT CONFIGURATIONS
%----------------------------------------------------------------------------------------

\documentclass[
	12pt, % Default font size, values between 10pt-12pt are allowed
	%letterpaper, % Uncomment for US letter paper size
	%spanish, % Uncomment for Spanish
]{fphw}

% Template-specific packages
\usepackage[utf8]{inputenc} % Required for inputting international characters
\usepackage[T1]{fontenc} % Output font encoding for international characters
\usepackage{mathpazo} % Use the Palatino font

\usepackage{graphicx} % Required for including images

\usepackage{booktabs} % Required for better horizontal rules in tables

\usepackage{listings} % Required for insertion of code

\usepackage{enumerate} % To modify the enumerate environment

\usepackage{float} %设置图片浮动位置的宏包
\usepackage{subfigure} %插入多图时用子图显示的宏包
\usepackage{ctex}
\usepackage{amsmath}
\usepackage{mathrsfs}

%----------------------------------------------------------------------------------------
%	ASSIGNMENT INFORMATION
%----------------------------------------------------------------------------------------

\title{Homework \#2} % Assignment title

\author{Chengqi Liu (1954148), Maitraiyi Dandekar (1990136),\\ Zakariae Jabbour (2039702)} % Student name

\date{September 19th, 2023} % Due date

\institute{Eindhoven University of Technology} % Institute or school name

\class{Cryptology (2MMC10)} % Course or class name

\professor{Tanja Lange} % Professor or teacher in charge of the assignment

%----------------------------------------------------------------------------------------

\begin{document}

\maketitle % Output the assignment title, created automatically using the information in the custom commands above

%----------------------------------------------------------------------------------------
%	ASSIGNMENT CONTENT
%----------------------------------------------------------------------------------------
\textbf{1. (a)\\}
The claim is true.\\
\textbf{Prove by contradiction:\\}
Assume that $H$ is not preimage resistant. So, given $\forall y$, the adversary can find $m'$ with non-negligible probability such that $H(k,m')=y$.\\
Now, given $z=h(k,m)$ as an image, we can just run the same algorithm, consider $z$ as an image of $H$ and find the preimage $x_1$. So, whe have
\[
	z=H(k,x_1)=h(k,h(k,x_1))
\]
We can just calculate $x_2=h(k,x_1)$, and get $z=h(k,x_2)$. So, $x_2$ is a preimage of $z$. So, $h$ is not preimage resistant. It's a contradiction.\\
So, If $h$ is preimage resistant, H is preimage resistant.\\
\textbf{(b)}\\
The claim is false.\\
Assume that $h_1$ is collision resistant but not preimage resistant, and $h_2$ is not collision resistant.\\
Because $h_2$ is not collision resistant, the adversary can find $y_1,y_2$ such that $h_2(k_2,y_1)=h_2(k_2,y_2)$. Then, because $h_1$ is not preimage resistant, the adversary can find $m_1,m_2$ such that $h_1(k_1,m_1)=y_1,h_1(k_1,m_2)=y_2$. In all, the adversary can find $m_1,m_2$ such that $h_2(k_2,h_1(k_1,m_1))=h_2(k_2,h_1(k_1,m_2))$, which is $(\langle k_1,k_2\rangle,m_1)=(\langle k_1,k_2\rangle,m_2)$. So, the adversary can find a collision of $H$ with non-negligible probability. It's a counterexample. The claim is wrong.\\
\\
\textbf{2. (a)}
\begin{align*}
	&s_1=r^{-1}(H(m_1)+x(R)a) \mod l\\
	&s_2=r^{-1}(H(m_2)+x(R)a) \mod l\\
	\Rightarrow&s_1s_2^{-1}=(H(m_1)+x(R)a)(H(m_2)+x(R)a)^{-1} \mod l\\
	\Rightarrow&s_1(H(m_2)+x(R)a)=s_2(H(m_1)+x(R)a) \mod l
\end{align*}
Notice that a is the only unknown variable in the last equation. So, we can solve the linear equation with one unknown and get
\[a=(s_2H(m_1)-s_1H(m_2))x(R)^{-1}(s_1-s_2)^{-1} \mod l\]
So, we can just compute $(s_2H(m_1)-s_1H(m_2))x(R)^{-1}(s_1-s_2)^{-1} \mod l$ to get $a$.\\
\textbf{(b)}
\begin{align*}
	&R_1=r_1P\\
	&R_2=(r_1+1)P\\
	&s_1=r_1^{-1}(H(m_1)+x(R_1)a) \mod l\\
	&s_2=(r_1+1)^{-1}(H(m_2)+x(R_2)a) \mod l
\end{align*}
Observe the last two equations. There are two unknown variables $r_1, a$, and there are also two equations. So, we can solve the linear equations with two unknowns and get
\[a=(s_1H(m_2)-s_2H(m_1)-s_1s_2)(s_2x(R_1)-s_1x(R_2))^{-1} \mod l\]
So, we can just compute $(s_1H(m_2)-s_2H(m_1)-s_1s_2)(s_2x(R_1)-s_1x(R_2))^{-1} \mod l$ to get $a$.\\
\textbf{(C)}\\
First, expand this equation:
\begin{align*}
	&w_1P+w_2P_A=R\\
	\Leftrightarrow& s^{-1}H(m)P+s^{-1}x(R)aP=rP \mod l\\
	\Leftrightarrow& s^{-1}(H(m)+x(R)a)P=rP  \mod l \\
	\Leftrightarrow& s^{-1}(H(m)+x(R)a)=r  \mod l
\end{align*}
(Note that we only need to modulo the factor multiplied with P with $l$. In order to write it beautifully, we wrote $\mod l$ at the end.)\\
So, for $m_1$ and $m_2$, we have
\begin{align*}
	&s_1^{-1}(H(m_1)+x(R_1)a)=r  \mod l \\
	&s_2^{-1}(H(m_2)+x(R_2)a)=r  \mod l
\end{align*}
Observe the last two equations. There are two unknown variables $r, a$, and there are also two equations. So, we can solve the linear equations with two unknowns and get
\[a=(s_1H(m_2)-s_2H(m_1))(s_2x(R_1)-s_1x(R_2))^{-1} \mod l\]
\end{document}

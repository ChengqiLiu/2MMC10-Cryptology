%%%%%%%%%%%%%%%%%%%%%%%%%%%%%%%%%%%%%%%%%
% fphw Assignment
% LaTeX Template
% Version 1.0 (27/04/2019)
%
% This template originates from:
% https://www.LaTeXTemplates.com
%
% Authors:
% Class by Felipe Portales-Oliva (f.portales.oliva@gmail.com) with template 
% content and modifications by Vel (vel@LaTeXTemplates.com)
%
% Template (this file) License:
% CC BY-NC-SA 3.0 (http://creativecommons.org/licenses/by-nc-sa/3.0/)
%
%%%%%%%%%%%%%%%%%%%%%%%%%%%%%%%%%%%%%%%%%

%----------------------------------------------------------------------------------------
%	PACKAGES AND OTHER DOCUMENT CONFIGURATIONS
%----------------------------------------------------------------------------------------


\documentclass[
12pt, % Default font size, values between 10pt-12pt are allowed
%letterpaper, % Uncomment for US letter paper size
%spanish, % Uncomment for Spanish
]{fphw}
% Template-specific packages
\usepackage[utf8]{inputenc} % Required for inputting international characters
\usepackage[T1]{fontenc} % Output font encoding for international characters
\usepackage{mathpazo} % Use the Palatino font

\usepackage{graphicx} % Required for including images

\usepackage{booktabs} % Required for better horizontal rules in tables

\usepackage{listings} % Required for insertion of code

\usepackage{enumerate} % To modify the enumerate environment

\usepackage{float} %设置图片浮动位置的宏包
\usepackage{subfigure} %插入多图时用子图显示的宏包
\usepackage{ctex}
\usepackage{amsmath}
\usepackage{mathrsfs}

\usepackage{listings}
\usepackage{color}

\definecolor{dkgreen}{rgb}{0,0.6,0}
\definecolor{gray}{rgb}{0.5,0.5,0.5}
\definecolor{mauve}{rgb}{0.58,0,0.82}

\lstset{frame=tb,
	language=Python,
	aboveskip=3mm,
	belowskip=3mm,
	showstringspaces=false,
	columns=flexible,
	basicstyle={\small\ttfamily},
	numbers=none,
	numberstyle=\tiny\color{gray},
	keywordstyle=\color{blue},
	commentstyle=\color{dkgreen},
	stringstyle=\color{mauve},
	breaklines=true,
	breakatwhitespace=true,
	tabsize=3
}
%----------------------------------------------------------------------------------------
%	ASSIGNMENT INFORMATION
%----------------------------------------------------------------------------------------

\title{Homework \#4} % Assignment title

\author{Chengqi Liu (1954148), Maitraiyi Dandekar (1990136),\\ Zakariae Jabbour (2039702)} % Student name

\date{October 3th, 2023} % Due date

\institute{Eindhoven University of Technology} % Institute or school name

\class{Cryptology (2MMC10)} % Course or class name

\professor{Tanja Lange} % Professor or teacher in charge of the assignment

%----------------------------------------------------------------------------------------

\begin{document}

\maketitle % Output the assignment title, created automatically using the information in the custom commands above

%----------------------------------------------------------------------------------------
%	ASSIGNMENT CONTENT
%----------------------------------------------------------------------------------------
\textbf{1.\\}
First, assume that the slope of the line is $l$. Calculate the addition and double formula of the elliptic curve.\\
Addition formula:
\begin{align*}
	(x_p,y_p)&+(x_q,y_q)=(x_r,y_r)\\
	l=&\frac{y_p-y_q}{x_p-x_q} \mod 41\\
	x_r=&l^2-x_p-x_q \mod 41\\
	y_r=&l(x_p-x_r)-y_p \mod 41
\end{align*}
Double formula:
\begin{align*}
	&2(x,y)=(x_r,y_r)\\
	l=&\frac{3x^2+a}{2y}=\frac{3x^2+1}{2y}\mod 41\\
	x_r=&l^2-2x \mod 41\\
	y_r=&l(x-x_r)-y \mod 41
\end{align*}
Run the code "Get\_R0\_to\_R3.py" to compute $R0\sim R3$.\\
(It should be placed in the same directory as file "Pollard\_rho.py" when running.)
\begin{lstlisting}
from Pollard_rho import kP_plus_mPA

if __name__=="__main__":
	p=53
	P_x=3
	P_y=38
	PA_x=25 
	PA_y=34
	print("W0: ",kP_plus_mPA(2,3,P_x,P_y,PA_x,PA_y))
	print("R0: ",kP_plus_mPA(23,13,P_x,P_y,PA_x,PA_y))
	print("R1: ",kP_plus_mPA(19,11,P_x,P_y,PA_x,PA_y))
	print("R2: ",kP_plus_mPA(2,41,P_x,P_y,PA_x,PA_y))
	print("R3: ",kP_plus_mPA(25,37,P_x,P_y,PA_x,PA_y))
\end{lstlisting}
The output is:\\
W0:  (20, 39)\\
R0:  (23, 22)\\
R1:  (30, 20)\\
R2:  (0, 26)\\
R3:  (27, 38)\\
So,
\begin{align*}
	W_0=&2P+3P_A=(20, 39)\\
	R_0=&23P+13P_A=(23, 22)\\
	R_1=&19P+11P_A=(30, 20)\\
	R2=&2P+41P_A=(0, 26)\\
	R3=&25P+37P_A=(27, 38)
\end{align*}
We define the 4 set $S_0,S_1,S_2,S_3$ as below:
\begin{align*}
	S_0=&\{x\in\mathbb{Z}_{41}:x\equiv0\mod41\}\\
	S_1=&\{x\in\mathbb{Z}_{41}:x\equiv1\mod41\}\\
	S_2=&\{x\in\mathbb{Z}_{41}:x\equiv2\mod41\}\\
	S_3=&\{x\in\mathbb{Z}_{41}:x\equiv3\mod41\}
\end{align*}
We define the step function $f(x,y)$ by $(x,y)+R_i$ when $i\equiv x(W)\mod4$:
\[ f(x,y)=\left\{
\begin{aligned}
	&(x,y)+(23,22),x\equiv0\mod4\\
	&(x,y)+(30, 20), x\equiv1\mod4 \\
	&(x,y)+(0, 26), x\equiv2\mod4\\
	&(x,y)+(27, 38), x\equiv3\mod4
\end{aligned}
\right.
\]
(You can find our codes in the file "Pollard\_rho.py". The codes are relatively long and is less readable if we paste them here.)\\
Run the program "Pollard\_rho.py" and get the output:\\
S0:(20,39) F0:(20,39) \\
S1:(21,35) F1:(11,3) \\
S2:(11,3) F2:(7,1)\\
S3:(23,22) F3:(34,30)\\
S4:(7,1) F4:(23,19)\\
S5:(30,20) F5:(3,3)\\
S6:(34,30) F6:(39,16)\\
S7:(9,26) F7:(16,14)\\
S8:(23,19) F8:(13,37)\\
S9:(11,38) F9:(39,25)\\
S10:(3,3) F10:(40,10)\\
S11:(29,24) F11:(23,19)\\
S12:(39,16) F12:(3,3)\\
S13:(0,15) F13:(39,16)\\
S14:(16,14) F14:(16,14)\\
The answer is: 23\\
The result is verified to be correct.
\end{document}

%%%%%%%%%%%%%%%%%%%%%%%%%%%%%%%%%%%%%%%%%
% fphw Assignment
% LaTeX Template
% Version 1.0 (27/04/2019)
%
% This template originates from:
% https://www.LaTeXTemplates.com
%
% Authors:
% Class by Felipe Portales-Oliva (f.portales.oliva@gmail.com) with template 
% content and modifications by Vel (vel@LaTeXTemplates.com)
%
% Template (this file) License:
% CC BY-NC-SA 3.0 (http://creativecommons.org/licenses/by-nc-sa/3.0/)
%
%%%%%%%%%%%%%%%%%%%%%%%%%%%%%%%%%%%%%%%%%

%----------------------------------------------------------------------------------------
%	PACKAGES AND OTHER DOCUMENT CONFIGURATIONS
%----------------------------------------------------------------------------------------


\documentclass[
12pt, % Default font size, values between 10pt-12pt are allowed
%letterpaper, % Uncomment for US letter paper size
%spanish, % Uncomment for Spanish
]{fphw}
% Template-specific packages
\usepackage[utf8]{inputenc} % Required for inputting international characters
\usepackage[T1]{fontenc} % Output font encoding for international characters
\usepackage{mathpazo} % Use the Palatino font

\usepackage{graphicx} % Required for including images

\usepackage{booktabs} % Required for better horizontal rules in tables

\usepackage{listings} % Required for insertion of code

\usepackage{enumerate} % To modify the enumerate environment

\usepackage{float} %设置图片浮动位置的宏包
\usepackage{subfigure} %插入多图时用子图显示的宏包
\usepackage{ctex}
\usepackage{amsmath}
\usepackage{mathrsfs}

\usepackage{listings}
\usepackage{color}

\definecolor{dkgreen}{rgb}{0,0.6,0}
\definecolor{gray}{rgb}{0.5,0.5,0.5}
\definecolor{mauve}{rgb}{0.58,0,0.82}

\lstset{frame=tb,
	language=Python,
	aboveskip=3mm,
	belowskip=3mm,
	showstringspaces=false,
	columns=flexible,
	basicstyle={\small\ttfamily},
	numbers=none,
	numberstyle=\tiny\color{gray},
	keywordstyle=\color{blue},
	commentstyle=\color{dkgreen},
	stringstyle=\color{mauve},
	breaklines=true,
	breakatwhitespace=true,
	tabsize=3
}
%----------------------------------------------------------------------------------------
%	ASSIGNMENT INFORMATION
%----------------------------------------------------------------------------------------

\title{Homework \#3} % Assignment title

\author{Chengqi Liu (1954148), Maitraiyi Dandekar (1990136),\\ Zakariae Jabbour (2039702)} % Student name

\date{September 26th, 2023} % Due date

\institute{Eindhoven University of Technology} % Institute or school name

\class{Cryptology (2MMC10)} % Course or class name

\professor{Tanja Lange} % Professor or teacher in charge of the assignment

%----------------------------------------------------------------------------------------

\begin{document}

\maketitle % Output the assignment title, created automatically using the information in the custom commands above

%----------------------------------------------------------------------------------------
%	ASSIGNMENT CONTENT
%----------------------------------------------------------------------------------------
\textbf{1.\\}
The curve $E$ is an Edwards Curve with $d=-5$. The following calculations are on $\mathbb{F}_{13}$:\\
Using the formula $(x_1,y_1)+(x_2,y_2)=(\frac{x_1y_2+x_2y_1}{1-5x_1x_2y_1y_2} \mod 13,\frac{y_1y_2-x_1x_2}{1+5x_1x_2y_1y_2}\mod 13)$,
\begin{align*}
	R=&2P+Q\\
	=&2(6,3)+(3,7)\\
	=&(6,10)+(3,7)\\
	=&(12,0)
\end{align*}
Edwards Curve is a kind of Twisted Edwards Curve with $a=1$. In $M:5v^2=u^3+3u^2+u$, we have
\begin{align*}
	A=&2\frac{a+d}{a-d}=3 \mod 13\\
	B=&\frac{4}{a-d}=5\mod 13
\end{align*}
So,
\[M:5v^2=u^3+3u^2+u\]
The formula of mapping from Edwards Curve $E$ to Montgomery Curve $M$ is $(x,y)\rightarrow (u,v)=(\frac{1+y}{1-y} \mod 13,\frac{1+y}{x(1-y)} \mod 13)$. So,
\begin{align*}
	P=&(6,3)\rightarrow P'=(11,4)\\
	Q=&(3,7)\rightarrow Q'=(3,1)\\
	R=&(12,0)\rightarrow R'=(1,12)
\end{align*}
Suppose the slope of a line through the two points is $l$. The double formula on $M$ is $2(u,v)=(u',v')$, where
\begin{align*}
	l=&\frac{3u^2+2Au+1}{2Bv}=\frac{3u^2+6u+1}{10v} \mod 13\\
	u'=&Bl^2-A-2u=5l^2-3-2u \mod 13\\
	v'=&l(u-u')-v \mod 13
\end{align*}
So, \[2P'=(6,1),\  \text{with}\ l=1\]
The addition formula on $M$ is $(u_1,v_1)+(u_2,v_2)=(u_3,v_3)$, where
\begin{align*}
	l=&\frac{v_2-v_1}{u_2-u_1} \mod 13\\
	u'=&Bl^2-A-u_1-u_2=5l^2-3-u_1-u_2 \mod 13\\
	v'=&l(u-u_1)-v_1 \mod 13
\end{align*}
So, \[2P'+Q'=(6,1)+(3,1)=(1,12)\  \text{with}\ l=0\]
So we get $2P'+Q'=R'$.\\\\
\textbf{2.}\\
First, assume that the slope of the line is $l$. Calculate the addition and double formula of the elliptic curve.\\
Addition formula:
\begin{align*}
	(x_p,y_p)&+(x_q,y_q)=(x_r,y_r)\\
	l=&\frac{y_p-y_q}{x_p-x_q} \mod 41\\
	x_r=&l^2-x_p-x_q \mod 41\\
	y_r=&l(x_p-x_r)-y_p \mod 41
\end{align*}
Double formula:
\begin{align*}
	&2(x,y)=(x_r,y_r)\\
	l=&\frac{3x^2+a}{2y}=\frac{3x^2+1}{2y}\mod 41\\
	x_r=&l^2-2x \mod 41\\
	y_r=&l(x-x_r)-y \mod 41
\end{align*}
You can also find our codes in the file "BSGS.py".
\begin{lstlisting}
import math
from Crypto.Util.number import inverse

a=1
b=20

def Double(x,y,modn):
	'''Double the point (x,y) modulo modn. Result is (x_r,y_r).'''
	if math.isinf(x) or math.isinf(y):
		return math.inf,math.inf
	l=(3*x*x+a)*inverse(2*y,modn)%modn
	x_r=(l*l-2*x)%modn
	y_r=(l*(x-x_r)-y)%modn
	return x_r,y_r

def Add(x_p,y_p,x_q,y_q,modn):
	'''Add two poins (x_p,y_p) and (x_q,y_q) modulp modn. Result is (x_r,y_r).'''
	if math.isinf(x_p):
		return x_q,y_q
	if math.isinf(x_q):
		return x_p,y_p
	if x_p==x_q and y_p!=y_q:
		return math.inf,math.inf
	if x_p==x_q and y_p==y_q:
		return Double(x_p,y_p,modn)
	l=(y_p-y_q)*inverse((x_p-x_q),modn)%modn
	x_r=(l*l-x_p-x_q)%modn
	y_r=(l*(x_p-x_r)-y_p)%modn
	return x_r,y_r

def Power(x,y,exp,modn):
	'''Calculate the exp times of the point (x,y) modulo modn. Result is (x_r,y_r).'''
	n=1
	x_r=x
	y_r=y
	while n<exp:
		x_r,y_r=Add(x_r,y_r,x,y,modn)
		n=n+1
	return x_r,y_r

def BSGS(x_A,y_A,x,y,order,modn):
	'''
	Solve DLP using Baby Step Giant Step. 
	Return log_{(x,y)}(x_A,x_A) modulo modn. 
	The point (x,y) has order "order".
	'''
	t=math.floor(math.sqrt(order))
	_xt,_yt=Power(x,y,t,modn)
	k=1
	_xkt=_xt
	_ykt=_yt
	g_map=dict()
	g_map[(_xt,_yt)]=k
	while k<math.floor(order/t):
		k=k+1
		_xkt,_ykt=Add(_xkt,_ykt,_xt,_yt,modn)
		g_map[(_xkt,_ykt)]=k
	i=1
	while i<t+1:
		_xi,_yi=Power(x,y,i,modn)
		_xAi,_yAi=Add(x_A,y_A,_xi,_yi,modn)
		if (_xAi,_yAi) in g_map:
			return (g_map[(_xAi,_yAi)]*t-i)%modn
		i=i+1
	return -1

if __name__=="__main__":
	modn=41
	p=53
	P_x=3
	P_y=38
	PA_x=25 
	PA_y=34
	print("The answer is:",BSGS(PA_x,PA_y,P_x,P_y,p,modn))
\end{lstlisting}
Run the program and get the output:\\
The answer is: 23
\end{document}
